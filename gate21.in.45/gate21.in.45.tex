\let\negthickspace\undefined
\documentclass[journal,12pt,twocolumn]{IEEEtran}
\usepackage{cite}
\usepackage{amsmath,amssymb,amsfonts,amsthm}
\usepackage{algorithmic}
\usepackage{graphicx}
\usepackage{textcomp}
\usepackage{xcolor}
\usepackage{txfonts}
\usepackage{listings}
\usepackage{enumitem}
\usepackage{mathtools}
\usepackage{gensymb}
\usepackage{comment}
\usepackage[breaklinks=true]{hyperref}
\usepackage{tkz-euclide} 
\usepackage{listings}
\usepackage{gvv}                                        
\def\inputGnumericTable{}                                 
\usepackage[latin1]{inputenc}                                
\usepackage{color}                                            
\usepackage{array}                                            
\usepackage{longtable}                                       
\usepackage{calc}                                             
\usepackage{multirow}                                         
\usepackage{hhline}                                           
\usepackage{ifthen}                                           
\usepackage{lscape}
\usepackage{tfrupee}

\newtheorem{theorem}{Theorem}[section]
\newtheorem{problem}{Problem}
\newtheorem{proposition}{Proposition}[section]
\newtheorem{lemma}{Lemma}[section]
\newtheorem{corollary}[theorem]{Corollary}
\newtheorem{example}{Example}[section]
\newtheorem{definition}[problem]{Definition}
\newcommand{\BEQA}{\begin{eqnarray}}
\newcommand{\EEQA}{\end{eqnarray}}
\newcommand{\define}{\stackrel{\triangle}{=}}
\theoremstyle{remark}
\newtheorem{rem}{Remark}
\begin{document}

\bibliographystyle{IEEEtran}
\vspace{3cm}

\title{Gate21.IN.45}
\author{EE23BTECH11062 - V MANAS}
\maketitle
\newpage

\bigskip
\textbf{Question:}\\A sinusoid $(\sqrt{2}sin(t))\mu(t)$,where $\mu(t)$ is the step input,is applied to a system with transfer function G(s)=$\frac{1}{1+s}$.The amplitude of the steady state output is\\
\textbf{Solution:}
\begin{table}[h]
    \centering
    \begin{tabular}{|c|c|c|c}
    \hline
    \textbf{Variable} & \textbf{Description} & \textbf{Values}\\
    \hline
    $G_o$ & overall transfer function & 1\\
    \hline
    $G_p$ & process transfer function & \\
    \hline
    $G_c$ & proportional controller transfer function & \\
    \hline
    $K_c$ & gain of the proportional controller & \\
    \hline
\end{tabular}

    \caption{Transformation}
    \label{tab:table_gate21.in.45}
\end{table}\\
By using laplace transform,
\begin{align}
    X(s)&=\frac{1}{1+s^2}\\
    Y(s)&=G(s)\times X(s)\\
    Y(s)&=(\frac{1}{1+s})(\frac{1}{1+s^2})
\end{align}
By finding inverse Laplace transform of Y(s),
\begin{align}
    y(t)=\frac{e^{-t}}{2}+\frac{1}{2}(sin(t)-cos(t))\\
    y(t)=\frac{e^{-t}}{2}-\frac{1}{\sqrt{2}}(sin(\frac{\pi}{4}-t))
\end{align}
So at steady state(at t$\rightarrow\infty$) the amplitude of output will be $\frac{1}{\sqrt{2}}$
\end{document}
