\let\negthickspace\undefined
\documentclass[journal,12pt,twocolumn]{IEEEtran}
\usepackage{cite}
\usepackage{amsmath,amssymb,amsfonts,amsthm}
\usepackage{algorithmic}
\usepackage{graphicx}
\usepackage{textcomp}
\usepackage{xcolor}
\usepackage{txfonts}
\usepackage{listings}
\usepackage{enumitem}
\usepackage{mathtools}
\usepackage{gensymb}
\usepackage{comment}
\usepackage[breaklinks=true]{hyperref}
\usepackage{tkz-euclide} 
\usepackage{listings}
\usepackage{gvv}                                        
\def\inputGnumericTable{}                                 
\usepackage[latin1]{inputenc}                                
\usepackage{color}                                            
\usepackage{array}                                            
\usepackage{longtable}                                       
\usepackage{calc}                                             
\usepackage{multirow}                                         
\usepackage{hhline}                                           
\usepackage{ifthen}                                           
\usepackage{lscape}
\usepackage{tfrupee}

\newtheorem{theorem}{Theorem}[section]
\newtheorem{problem}{Problem}
\newtheorem{proposition}{Proposition}[section]
\newtheorem{lemma}{Lemma}[section]
\newtheorem{corollary}[theorem]{Corollary}
\newtheorem{example}{Example}[section]
\newtheorem{definition}[problem]{Definition}
\newcommand{\BEQA}{\begin{eqnarray}}
\newcommand{\EEQA}{\end{eqnarray}}
\newcommand{\define}{\stackrel{\triangle}{=}}
\theoremstyle{remark}
\newtheorem{rem}{Remark}
\begin{document}

\bibliographystyle{IEEEtran}
\vspace{3cm}

\title{11.9.5.3}
\author{EE23BTECH11062 - V MANAS}
\maketitle
\newpage

\bigskip
\textbf{Question:}\\Let the sum of $n,2n,3n$ terms of an AP be $S_1,S_2$ and $S_3$, respectively, show that $S_3=3(S_2-S_1)$\\
\textbf{Solution:}
\begin{table}[h]
    \centering
    \begin{tabular}{|c|c|c|}
    \hline
    \textbf{Variable} & \textbf{Description}\\
    \hline
    x(0) & First term of AP\\
    \hline
    d & common difference in the AP\\
    \hline
    n & number of terms in AP\\
    \hline
\end{tabular}

    \caption{Variables Used}
    \label{tab:table_11.9.5.3}
\end{table}\\
By performing inverse Z transform on $S_1$(z) using contour integration
\begin{align}
    S_1&=\frac{1}{2{\pi}j}{\oint_c}S\brak{Z}z^{n-1}dz\\
    S_1&=\frac{1}{2{\pi}j}{\oint_c}\brak{\frac{x(0)z^{n-1}}{\brak{1-z^{-1}}^2}+\frac{dz^{n-2}}{\brak{1-z^{-1}}^3}}dz
\end{align}
For R1 the pole has been repeated twice
\begin{align}
    R_1&=\frac{1}{1!}\lim_{z\to 1}\frac{d}{dz}\brak{(z-1)^2\frac{x(0)z^{n+1}}{(z-1)^2}}\\
    &=x(0)(n+1)\lim_{z\to 1}z^n\\
    &=x(0)(n+1)
\end{align}
For R2 the pole has been repeated thrice
\begin{align}
    R_2&=\frac{1}{2!}\lim_{z\to 1}\frac{d^2}{dz^2}\brak{(z-1)^3\frac{dz^{n+1}}{(z-1)^3}}\\
    &=\frac{d(n+1)}{2}\lim_{z\to 1}\frac{d}{dz}(z^n)\\
    &=\frac{d(n+1)(n)}{2}\lim_{z\to 1}(z^{n-1})\\
    &=\frac{d(n+1)(n)}{2}
\end{align}
$\implies R=R_1+R_2$
\begin{align}
    \implies S_1=\frac{n+1}{2}(2x(0)+nd)u(n)
\end{align}
similarly,
\begin{align}
    \implies S_2=\frac{2n+1}{2}(2x(0)+2nd)u(n)\\
    \implies S_3=\frac{3n+1}{2}(2x(0)+3nd)u(n)
\end{align}
$\implies RHS=3(S_2-S_1)$
\begin{align}
    RHS&=3(\frac{2n+1}{2}(2x(0)+2nd)u(n)-\frac{n+1}{2}(2x(0)+nd)u(n))\\
    &=\frac{3n+1}{2}(2x(0)+3nd)u(n)
\end{align}
$\therefore$ LHS=RHS shows that $S_3=3(S_2-S_1)$

\end{document}
